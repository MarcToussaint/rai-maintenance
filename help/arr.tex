\documentclass[11pt,fleqn,twoside]{article}
\usepackage{amsmath}
\usepackage{amssymb}
\usepackage{amsfonts}
\usepackage{amsthm}

\usepackage{geometry}
\geometry{a4paper,hdivide={20mm,*,20mm},vdivide={20mm,*,20mm}}
\renewcommand{\arraystretch}{1.1}
\parindent 0pt
\parskip 3mm

\usepackage{fancyvrb}
\DefineShortVerb{\@}
\fvset{numbers=none,xleftmargin=5ex,fontsize=\footnotesize}

\pagestyle{empty}

\input{macros}

%  \renewcommand{\familydefault}{\sfdefault}
\usepackage{color}
\usepackage{graphicx}
\pdflatex

%%%%%%%%%%%%%%%%%%%%%%%%%%%%%%%%%%%%%%%%%%%%%%%%%%%%%%%%%%%%%%%%%%%%%%%%%%%%%%%%

\begin{document}
%\maketitle
\textbf{Tensor notations for rai::arr}\hfill\emph{Marc Toussaint}

The @rai::Array@ class a standard tensor storage, wrapping
@std::Vector@. It's implementation and interface is very similar to
Octave's Array class. I just provide my recommentation for C++
operator overloading for easy linear algebra syntax:

\noindent
\begin{tabular}{|p{.2\columnwidth}|p{.35\columnwidth}|p{.35\columnwidth}|}
\hline
& Tensor/Matlab notation
& C++
\\
\hline
``inner'' product\footnotemark[1]
& $C_{ijl} = \sum_k A_{ijk} B_{kl}$
& @A * B@
\\
index-wise product\footnotemark[2]
& $c_i = a_i b_i$ ~or~ $c=a\circ b$
& @a % b@
\\
& $C_{ijkl} = A_{ijk} B_{kl}$
& @A % B@
\\
diag
& $\diag(a)$
& @diag(a)@
\\
& $\diag(a) B$ ~or~ $c_i = a_i B_{ij}$
& @a % B@ ~or~ @diag(a) * B@
\\
element-wise product
& $c_i = a_i b_i$ ~or~ $c=a\circ b$
& @a % b@
\\
& $C_{ij} = A_{ij} B_{ij}$  ~or~ $C=A\circ B$
& no operator-overload!\footnotemark[3]\newline @elemWiseProduct(A,B)@
\\
outer product
& $C_{ijklm} = A_{ijk} B_{lm}$
& @A ^ B@
\\
& $a b^\T$ ~ (vectors)
& @a ^ b@
\\
transpose
& $A_{ij} = B_{ji}$
& @A = ~B@
\\
inverse
& $A B^\1 C$
& @A*(1/B)*C@
\\
& $A^\1 b$ ~ (or @A\b@ in Matlab)
& @A|b@ %(\footnotemark[4])
\\
\hline
element \emph{reference}\footnotemark[5]
& $A_{103}$
& @A(1,0,3)@
\\
& $A_{(n-2)03}$
& @A(-2,0,3)@
\\
sub-\emph{references!}\footnotemark[6]
& $x_i=A_{2i}$
& @A(2,{})@ ~or~ @A[2]@
\\
& $C_{i} = A_{20i}$ ~or~ @C=A[2,0,:]@
& @A(2,0,{})@ ~ (\footnotemark[7])
\\
&  $C_{ijk} = A_{20ijk}$
& @A(2,0,{})@ ~ {\tiny (trailing @{},{}..@ are implicit)}
\\
sub-refercing ranges\footnotemark[8]
& @C=A[2:4,:,:]@
& @A({2,4})@ ~ {\tiny (trailing @{},{}..@ are implicit)}
\\
& @A[2,1:3,:]@
& @A(2,{1,3})@
\\
\hline
sub-\emph{copies}\footnotemark[9]
& @A(1:3, :, 5:)@
& @A.sub(1,3, 0,-1, 5,-1)@
\\
sub-selected-copies
& @A[{1,3,4},:,{2,3},2:5]@
& @A.sub({1,3,4}, 0,-1, {2,3}, 2,5)@
\\
\hline
sub-assignment\footnotemark[10]
& @A[4:6, 2:5] = B@ ~ ($B\in\RRR^{3\times 4}$)
& @A.setBlock(B, 4, 2)@
\\
& @x[4:6] = b@ ~ ($b\in\RRR^3$)
& @x.setBlock(b, 4)@ ~or~ @x({4,6}) = b@
\\
\hline
initialization
& @A=[1 2 3]'@
& @arr A={1.,2,3}@ ~or~ @arr A(3, {1.,2,3})@
\\
& @A=[1 2 3]@
& @arr A=~arr({1.,2,3})@ ~or~ @arr A(1, 3, {1.,2,3})@
\\
& @A=[1 2; 3 4]@
& @arr A(2,2, {1.,2,3,4})@
\\
\hline
concatenation
& $(x^\T,y^\T,z^\T)^\T$ ~ (stacked vectors)
& @(x,y,z)@
\\
& @cat(1,A,B)@ ~ (stacked matrices)
& @(A,B)@ ~ (memory serial)
\\
\hline
\end{tabular}
\footnotetext[1]{The word ``inner'' product should,
  strictly, be used to refer to a general 2-form $\<\cdot,\cdot\>$
  which, depending on coordinates, may have a non-Euclidean metric
  tensor. However, here we use it in the sense of ``assuming Euclidean
  metric''.}
\footnotetext[2]{For matrices or tensors, this is \emph{not} the
  element-wise (Hadamard) product!}
\footnotetext[3]{The elem-wize product for
  matricies/tensors is much less used within equations than what I call 'index-wise product'.}
%% \footnotetext[4]{Caution! This is notation is somewhat awkward -- it calls a special solver for $A^\1 b$
%%   instead of computing $A^\1$ separately.}
\footnotetext[5]{Negative indices are always interpreted as $n-index$. As
we start indexing from 0, the index $n$ is already out of range. $n-1$
is the last entry. An index of -1 therefore means 'last'.}
\footnotetext[6]{In C++, assuming the last index to be
  memory aligned, sub-referencing is only efficient w.r.t.\ major
  indices: the reference then points to the same memory as the parent
  tensor.}
\footnotetext[7]{As a counter example, \texttt{A[:,0,2]} could be referenced in a memory aligned manner. It can only be copied with \texttt{A.sub(0,-1,0,0,2,2)}}
\footnotetext[8]{again, this can only be ranges
  w.r.t.\ the major index, to ensure memory alignment}
%% \footnotetext[1]{Note: The same for \emph{copying} would be
%%   @A.sub(2,4,0,-1,0,-1)@ resp.\ @A.sub(2,2,1,3,0,-1)@ }
\footnotetext[9]{this is a $\RRR^{3\times ...}$
  matrix: The '3' is \emph{included} in the range.}
\footnotetext[10]{As the assignments are not memory-aligned,
  they can't be done with returned references.}

\end{document}
